\chapter*{Abstract}
\addcontentsline{toc}{chapter}{Abstract}

% Deep learning has shown great promise in recent years for various non-medical
% and medical problems. However, the size of images has been a burden and problems
% had to be mapped to either small 2D or 3D patches, or 2D images. This thesis
% analysis the potential of deep learning to solve medical image analysis problems
% using entire 3D volumes. A major part of the thesis was the development of a
% training method for deep learning models that facilitates the training on entire
% 3D volumes. We have applied the new training method to solve different medical
% image analysis problems, such as manifold learning for biomarker discovery for
% AD using raw images, and MS using deformation fields, and lesion masks. A second
% part of the thesis was the development of an MS lesion segmentation method using
% multimodal MR images. We evaluated the runtime improvements against other
% state-of-the-art methods. Furthermore, we showed the clinical value of our
% approach on medical problems.

% World limit: 350

% Context scope
% Problem and why is it challenging

\Gls{ms} is a inflammatory and degenerative disease of the central nervous
system characterized by the formation of lesions and axonal loss leading to
regional and global atrophy. Imaging biomarkers based on the delineation of
lesions, such as lesion load and lesion count, have established their importance
for assessing disease progression and treatment effect. However, lesions vary
greatly in size, shape, intensity and location, which makes their automatic and
accurate segmentation challenging. Furthermore, the current established imaging
biomarkers largely capture volumetric changes, which are important and
relatively easy to compute, but do not reflect potentially important structural
variations, such as shape changes in the brain and spatial dispersion of the
lesions.
% It would be highly desirable to have a method that can automatically
% discover potentially important patterns of variability in brain morphology and
% lesion distribution, which would advance our understanding of the complex
% pathology of MS.

% What is the goal of the thesis and how were they achieved

We propose a novel segmentation approach for MS lesions based on a neural
network that consists of two interconnected pathways for feature extraction and
lesion prediction, which allows for the automatic learning of features at
different scales that are optimized for accuracy for any given combination of
image types and segmentation task. We have evaluated our method on publicly
available data sets and a proprietary data set from a large MS clinical trial,
with the results showing that our method performs comparably to the top-ranked
methods on the public data sets, and outperforms five freely available and
widely used methods when a sufficiently large data set is available.

% . In addition, we have compared our method with five
% freely available and widely used MS lesion segmentation methods on a large data
% set from an MS clinical trial. The results show that our method consistently
% outperforms these other methods across a wide range of lesion sizes.

To further our understanding of complex MS pathology, we build a statistical
model of brain images that can automatically discover patterns of variability in
brain morphology and lesion distribution. We propose building such a model using
a deep belief network, a layered network whose parameters can be learned
from training images.
%  In contrast to other manifold learning algorithms, the DBN approach does not
% require a prebuilt proximity graph, which is particularly advantageous for
% modeling lesions, because their sparse and random nature makes defining a
% suitable distance measure between lesion images challenging.
Our results show that this model can automatically discover the classic patterns
of MS pathology, as well as more subtle ones, and that the parameters computed
have strong relationships to MS clinical scores.

Deep learning has traditionally been too computationally expensive for
application to 3D images due to the large number of trainable parameters. In
this thesis, we propose a much more computationally efficient training method
for deep convolutional models that makes training on 3D medical images with high
resolution practical.

\begin{comment}

% What are the main results

We propose a novel segmentation approach based on deep 3D convolutional encoder
networks with shortcut connections and apply it to the segmentation of multiple
sclerosis (MS) lesions in magnetic resonance images. Our model is a neural
network that consists of two interconnected pathways, a convolutional pathway,
which learns increasingly more abstract and higher-level image features, and a
deconvolutional pathway, which predicts the final segmentation at the voxel
level. The joint training of the feature extraction and prediction pathways
allows for the automatic learning of features at different scales that are
optimized for accuracy for any given combination of image types and segmentation
task. In addition, shortcut connections between the two pathways allow high- and
low-level features to be integrated, which enables the segmentation of lesions
across a wide range of sizes. We have evaluated our method on two publicly
available data sets (MICCAI 2008 and ISBI 2015 challenges) with the results
showing that our method performs comparably to the top-ranked state-of-the-art
methods, even when only relatively small data sets are available for training.
In addition, we have compared our method with five freely available and widely
used MS lesion segmentation methods (EMS, LST-LPA, LST-LGA, Lesion-TOADS, and
SLS) on a large data set from an MS clinical trial. The results show that our
method consistently outperforms these other methods across a wide range of
lesion sizes.

Manifold learning of medical images plays a potentially important role for
modeling anatomical variability within a population with applications that
include segmentation, registration, and prediction of clinical parameters. This
paper describes a novel method for learning the manifold of 3D brain images
that, unlike most existing manifold learning methods, does not require the
manifold space to be locally linear, and does not require a predefined
similarity measure or a prebuilt proximity graph. Our manifold learning method
is based on deep learning, a machine learning approach that uses layered
networks (called deep belief networks, or DBNs) and has received much attention
recently in the computer vision field due to their success in object recognition
tasks. DBNs have traditionally been too computationally expensive for
application to 3D images due to the large number of trainable parameters. Our
primary contributions are 1) a much more computationally efficient training
method for DBNs that makes training on 3D medical images with a resolution of up
to \num{128x128x128} practical, and 2) the demonstration that DBNs can learn a
low-dimensional manifold of brain volumes that detects modes of variations that
correlate to demographic and disease parameters.

Changes in brain morphology and white matter lesions are two hallmarks of
multiple sclerosis (MS) pathology, but their variability beyond volumetrics is
poorly characterized. To further our understanding of complex MS pathology, we
aim to build a statistical model of brain images that can automatically discover
spatial patterns of variability in brain morphology and lesion distribution. We
propose building such a model using a deep belief network (DBN), a layered
network whose parameters can be learned from training images. In contrast to
other manifold learning algorithms, the DBN approach does not require a prebuilt
proximity graph, which is particularly advantageous for modeling lesions,
because their sparse and random nature makes defining a suitable distance
measure between lesion images challenging. Our model consists of a morphology
DBN, a lesion DBN, and a joint DBN that models concurring morphological and
lesion patterns. Our results show that this model can automatically discover the
classic patterns of MS pathology, as well as more subtle ones, and that the
parameters computed have strong relationships to MS clinical scores.

Deep learning has traditionally been computationally expensive and advances in
training methods have been the prerequisite for improving its efficiency in
order to expand its application to a variety of image classification problems.
In this paper, we address the problem of efficient training of convolutional
deep belief networks by learning the weights in the frequency domain, which
eliminates the time-consuming calculation of convolutions. An essential
consideration in the design of the algorithm is to minimize the number of
transformations to and from frequency space. We have evaluated the running time
improvements using two standard benchmark datasets, showing a speed-up of up to
8 times on 2D images and up to 200 times on 3D volumes.
Our training algorithm makes training of convolutional deep belief networks on
3D medical images with a resolution of up to \num{128x128x128} voxels practical,
which opens new directions for using deep learning for medical image analysis.

\end{comment}
