%\svnid{$Id: titlepage.tex 43 2013-05-16 16:59:40Z tbrosch $}
\begin{titlepage}
\newlength{\smalltitlespace}
\newlength{\mediumtitlespace}
\newlength{\largetitlespace}
\setlength{\smalltitlespace}{1em}
\setlength{\mediumtitlespace}{1.5em}
\setlength{\largetitlespace}{3.5em}
\noindent
  \begin{center}
  \mbox{}\vfill
  \large\sffamily
  {\Large\bfseries\MakeUppercase{\longtitle}\\}\vspace{2em}
  by\\[1.5em]
  {\large TOM BROSCH\\[\smalltitlespace]}
  {\large Dipl.-Ing., Otto-von-Guericke-Universit\"at Magdeburg,
  2010\\[\largetitlespace]}
  {\large A THESIS SUBMITTED IN PARTIAL
  FULFILLMENT OF\\
  THE REQUIREMENTS FOR THE DEGREE OF\\[\smalltitlespace]
  DOCTOR OF PHILOSOPHY\\[\smalltitlespace]}
  in\\[\smalltitlespace]
  {\large THE FACULTY OF GRADUATE AND POSTDOCTORAL STUDIES\\[\smalltitlespace]
  (Biomedical Engineering)\\[\largetitlespace]}
  {\large THE UNIVERSITY OF BRITISH COLUMBIA\\[\smalltitlespace]
  (Vancouver)\\[\largetitlespace]}
  \thesisdate\today\\[\smalltitlespace]
  {\rmfamily \textcopyright{}} Tom Brosch, \thesisyear\today
  %\vfill
  \end{center}
\end{titlepage}
%\chapter*{Abstract}

%\mbox{}\vspace{2em}
% \begin{quotation}
% \begin{center}
% \textbf{Abstract}
% \end{center}
% \noindent
% Deep learning has shown great promise in recent years for various non-medical
% and medical problems. However, the size of images has been a burden and problems
% had to be mapped to either small 2D or 3D patches, or 2D images. This thesis
% analysis the potential of deep learning to solve medical image analysis problems
% using entire 3D volumes. A major part of the thesis was the development of a
% training method for deep learning models that facilitates the training on entire
% 3D volumes. We have applied the new training method to solve different medical
% image analysis problems, such as manifold learning for biomarker discovery for
% AD using raw images, and MS using deformation fields, and lesion masks. A second
% part of the thesis was the development of an MS lesion segmentation method using
% multimodal MR images. We evaluated the runtime improvements against other
% state-of-the-art methods. Furthermore, we showed the clinical value of our
% approach on medical problems.
% \end{quotation}
% \vfill
% \noindent
% \textbf{Brosch, Tom:}\\
% \emph{\longtitle}\\
% PhD thesis, The University of British Columbia, 2015