\chapter{Conclusions and Future Work}

In this thesis, we have addressed two challenges for measuring the state and
progression of MS. In \ref{sec:segmentation}, we presented a fully automatic
segmentation method of MS lesions, which enables the accurate computation of
lesion-based imaging biomarkers such as lesion load and lesion count. In
\ref{sec:manifold}, we presented a novel method for the automatic discovery of
patterns of variability in MS subjects that show improved correlations to
clinical scores over traditional volumetric biomarkers. Both methods are based
on deep learning, a class of algorithms that can learn a feature hierarchy from
a set of images. A major challenge of deep learning methods is the time and
memory required to process large 3D volumes. To address this challenge, we have
developed a novel training algorithm for convolutional deep learning methods
that facilitates the training of full resolution 3D medical images and presented
our algorithm with a comprehensive comparison with alternative training
methods in \ref{sec:training}. Although the developed methods were
developed and evaluated in the context of MS, the methods are very general and
can potentially be applied to various segmentation and manifold learning
problems.

\section{Contributions}

In the course of developing deep learning-based methods for MS lesion
segmentation and pattern discovery, we have made the following contributions:
\begin{itemize}
\item We have developed a novel training algorithm for convolutional deep belief
networks and convolutional neural networks that performs training in the
frequency domain. The speed-ups gained by our method compared to spatial domain
implementations and the reduced memory requirements compared to other frequency
domain methods facilitate the application of deep learning to high-resolution
3D medical images.
  
\item We have developed a neural network architecture that jointly learns the
features required to segment MS lesions and to perform the segmentation based on
the automatically learned features. The joint learning of feature extractor and
classifier facilitates the learning of features that are robust to the
variability of MS lesions and varying contrasts produced by different scanners.

\item We have developed a novel objective function for training neural networks
that is suitable for the classification of vastly unbalanced classes, such as
the segmentation of MS lesions, which typically comprise less than one percent
of the image.

\item We have demonstrated that the use of shortcut connections in the proposed
neural network architecture allows for the learning of features at different
scales, which facilitates the segmentation of both very small and very large
lesions.

\item We have developed a framework for modelling changes in brain morphology
and lesion distribution with only a few parameters, which also show improved
correlation with clinical scores compared to established volumetric imaging
biomarkers. In contrast to previous manifold learning approaches, our method
does not assume that the ambient space is locally linear and also does not
require the definition of a suitable similarity measure, or building a proximity
graph.
\end{itemize}

\section[Future work]{Future Work}

\subsection[New Applications of deep learning]{New Applications of Deep
Learning}

Motivated by the success of deep learning for the two applications explored in
this theses, a possible direction for further research is to investigate new
applications of deep learning. The success of deep learning methods mostly stems
from their ability to learn feature hierarchies directly from the data. This
allows deep learning methods to adapt to the challenges presented by the data,
such as imaging artifacts and contrast variations, without explicitly having to
model them. However, deep learning methods can take a lot of time to train and
require large amounts of training data in order to learn a feature
representation that generalizes well to new data samples. To overcome those
challenges, first applications of deep learning were limited to problems that
can be cast as a patch-wise classification problem, such as the segmentation of
cell membranes \citep{ciresan2012} and the detection of cell carcinoma
\citep{cruz2013}. In this fashion, every patch of an image is considered a
training sample, which greatly increases the amount of labelled training cases.
In addition, training on relatively small 2D patches made training feasible.
Similarly, fully convolutional approach for image segmentation leverage every
pixel of an image as a training sample, while being more computationally
efficient than patch-based approaches. However, layers deeper in the hierarchy
have increasingly fewer pixels, and are consequently trained with fewer training
samples, which increases the risk of overfitting.

An interesting research question is therefore what other problems can be solved
using deep learning. The proposed segmentation framework can readily be applied
to various segmentation problems. Preliminary results on the segmentation of the
corpus callosum show great promise. For future work, it will be important to
explore other segmentation problems. Another potential application could be the
detection of landmark points. Detecting landmarks is similar to segmentation in
that every voxel of an image can be leveraged as a training sample. However, the
number of positive samples per image is much lower than for segmentation, which
might require further research. A potential solution to deal with the sparseness
of a landmark map is to train the model on a distance transformation of the
original landmark map.

A different stream of potential applications is to incorporate deep learning
into existing model learning frameworks. Originally, active shape
\cite{cootes1995} and active appearance \cite{cootes2001q} models employ
principal component analysis in order to reduce the dimensionality of the input
feature space. Similarly, deep belief networks or stacked autoencoders can be
used for dimensionality reduction, which might yield to the learning of more
biologically plausible models, due to their ability to find highly nonlinear
patterns of variability in the data.

\subsection[Dealing with small data set using data augmentation]{Dealing with
Small Data Set using Data Augmentation}

One way of dealing with the problem of limited training data is data
augmentation. Data augmentation allows the artificial increase of the training
data set size by generating new training samples from existing ones. Data
augmentation has been successfully be used for to deal with the problem of small
medical data sets. For example, \citet{ronneberger2015} used automatically
generated deformations to artificially increase the data set size for segmenting
cells, but also simpler data augmentation techniques such as adding rotations,
translations and contrast variations can help to avoid overfitting. However,
hand-engineered data augmentation approaches depend on how well the variability
in the data is understood. Alternatively, automatically learned unsupervised
deep generative models such as deep belief networks can potentially be used to
learn the variability of the training data in order to generate new samples that
are biologically plausible.

\subsection[Rotation-invariant neural networks]{Rotation-invariant Neural
Networks}

Although neural networks are able to learn features that are invariant to the
variability in the data, doing so often requires the learning of different
variants of the same feature in order to capture the range of variability. In
order to decrease the number of weights of a neural network, and therefore to
reduce the training time and risk of overfitting, different neural network
architectures have been developed with build-in invariance or equivariance
properties. For example, previously used sigmoid activation functions are
sensitive to the intensity range of the input, which requires the relearning of
feature detectors for images with varying contrast. To overcome, rectified
linear units were proposed, which represent the same units with different
biases, which makes them equivariant to intensity variations. In order to allow
the learning of translational invariant features, convolutional neural networks
were developed, which greatly reduce the number of weights required to capture
simple edge detecting features at different locations in the image compared to
dense neural networks. However, features learned by a convolutional neural
network are not invariant to rotations. This leads, for example, to the learning
of a variety of edge detecting features that are sensitive to different angles.
While the number of angles is rather low in 2D, significantly more features are
required to span the space of rotations in 3D, due to the combination of three
rotation parameters. In order to avoid the learning of very basic features with
different rotation parameters in 3D, it might therefore be necessary to develop
neural network architectures that are inherently invariant or equivariant to
rotations, which would allow for the learning of a diverse set of features that
capture fine details in 3D with much fewer weights than currently employed
convolutional neural network architectures.

\subsection[Increasing the expressive power of single neurons]{Increasing the
Expressive Power of Single Neurons}

Current neural networks can only calculate the weighted sum of its inputs
followed by the application of a nonlinear function. Although it has been shown
that this model is able to approximate any function, doing so might require a
lot of units and layers, which makes them harder to train and requires more
training data in order to reduce the risk of overfitting. One way to increase
the expressive power of neural networks without increasing their complexity is
to design more computationally powerful units. In their review of the
computational properties of dendrites, \citet{london2005} noted that some
neurons were found that are able to calculate the multiplication of its inputs.
For example, the looming sensitive neurons in the locust's visual system are
able to detect approaching objects on a collision course by calculating the
multiplication of angular size and speed of approaching object. The
multiplication is possible encoded as the sum of logarithmic inputs followed by
exponentiation. Similarly, \citet{godfrey2016} have proposed a type of neuron
for artificial neural networks that is able to learn to calculate the logarithm
or exponential of its inputs, which allows for the learning of networks that can
calculate arbitrary polynomial with significantly fewer parameters than
traditional ReLU or sigmoid neural networks. Alternatively, \citet{livni2013}
have proposed a network architecture that can learn arbitrary polynomials using
a combination of units that calculate weighted sums of an arbitrary number of
inputs and units that calculate the product of two inputs. However, these
network architectures needs to be investigated further in order to show of the
theoretical advantages can be translated into improvements on practical
problems.
