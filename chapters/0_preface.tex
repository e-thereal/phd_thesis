\chapter*{Preface}
\addcontentsline{toc}{chapter}{Preface}

This thesis is primarily based on two journal papers, three conference
papers, and one book chapter, resulting from the collaboration between multiple
researchers. In all publications, the contribution of the author was in
developing, implementing, and evaluating the method. All co-authors
contributed to the editing of the manuscripts.
\\[1em]
Parts of the introduction to deep learning in \ref{sec:background} has been
published in:
\begin{itemize}
\item Brosch, Tom, Y. Yoo, L.Y.W. Tang, and R. Tam.
Deep learning of brain images and its application to multiple sclerosis.
In: G. Wu et al. (eds.): Machine Learning in Medical Imaging, pages 1--25,
2016.
\end{itemize}
The contribution of the author was in writing the introduction of the book
chapter. Y. Yoo and R. Tam wrote the remaining chapters. All co-authors
contributed to the editing of the manuscript.
\\[1em]
A study described in \ref{sec:training} has been published in:
\begin{itemize}
\item Brosch, Tom and R. Tam. Efficient training of convolutional deep
belief networks in the frequency domain for application to high-resolution 2D
and 3D images. Neural Computation, 27(1), pages 211--227, 2015.
\end{itemize}
The contribution of the author was in developing, implementing, and evaluating
the method. R. Tam helped with his valuable suggestions in improving the
methodology.
\\[1em]
A study described in \ref{sec:manifold} has been published in:
\begin{itemize}
\item Brosch, Tom, R. Tam, for the ADNI. Manifold learning of brain MRIs
by deep learning. In: K. Mori et als. (eds.): MICCAI 2013, Part II, LNCS 8150,
pages 633--640, 2013.
\end{itemize}
The contribution of the author was in developing, implementing, and evaluating
the method. R. Tam helped with his valuable suggestions in improving the
methodology.
\\[1em]
A study described in \ref{sec:manifold} has been published in:
\begin{itemize}
\item Brosch, Tom, Y. Yoo, A. Traboulsee, D.K.B. Li, and R. Tam. Modeling the
variability in brain morphology and lesion distribution in multiple sclerosis by
deep learning. In: P. Golland et al. (eds.): MICCAI 2014, Part II, LNCS 8674,
pages 462--469, 2014.
\end{itemize}
The contribution of the author was in developing, implementing, and evaluating
the method. A. Traboulsee and D.K.B. Li collected the data sets. Y. Yoo and R.
Tam helped with their valuable suggestions in improving the methodology.
\\[1em]
A study described in \ref{sec:segmentation} has been published in:
\begin{itemize}
\item Brosch, Tom, Y. Yoo, L.Y.W. Tang, D.K.B. Li, A. Traboulsee, and R. Tam.
Deep convolutional encoder networks for multiple sclerosis lesion segmentation.
In: N. Navab et al. (eds.): MICCAI 2015, Part III, LNCS 9351, pages 3--11, 2015.
\item Brosch, Tom, L.Y.W. Tang, Y. Yoo, D.K.B. Li, A. Traboulsee, and R. Tam.
Deep 3D convolutional encoder networks with shortcuts for multiscale feature
integration applied to multiple sclerosis lesion segmentation. IEEE Transactions
on Medical Imaging, 2016, \emph{accepted}.
\end{itemize}
The contribution of the author was in developing, implementing, and evaluating
the method. A. Traboulsee and D.K.B. Li collected the data sets. Y. Yoo, L.Y.W.
Tang, and R. Tam helped with their valuable suggestions in improving the
methodology.
