\chapter{Introduction}

% What is MS

\section{Motivation}

\subsection[A short introduction to multiple sclerosis]{A Short Introduction to
Multiple Sclerosis}

\Gls{ms} is a chronic and degenerative disease of the \gls{cns} characterized by
the formation of inflammatory and demyelinating lesions. Presumably due to the
breakdown of the blood-brain-barrier, the body's own immune system attacks the
myelin sheaths that act as insulating covers of the axons of neurons (see
\ref{fig:ms}). This disrupts the ability of parts of the nervous system to
communicate, and can lead to the manifestation of a large range of different
signs and symptoms. The brain is a very plastic organ and is often able to
compensate for the damage by forming new neural pathways. In the later stages of
the diseases, however, the amount of tissue damage is often too high to be
compensated, which leads to the progressive accumulation of disease.

\begin{figure}[tb]
\centering
\begin{tikzpicture}
\node[] (healthy)
{\includegraphics[width=0.5\textwidth]{figures/healthyneuron}};
\node[,below=-1.5cm of healthy] (damaged)
{\includegraphics[width=0.5\textwidth]{figures/damagedneuron}};

\node[fit=(healthy)(damaged), inner sep=0] (neurons) {};

\node[right=1cm of neurons] (brain)
{\includegraphics[width=0.4\textwidth]{figures/brain}};

\coordinate (m1) at (-0.75,0.9);
\coordinate (m2) at (0.95,0.9);
\coordinate (m3) at (2.4,-3);

\node[fit=(m1)(m2)] (myelin) {};

\node[above=0.65cm of myelin] (sheaths) {Myelin sheaths};

\draw (m1)--(sheaths);
\draw (m2)--(sheaths);

\node[ellipse,draw=red,thick,minimum width=1.5cm, minimum height=0.9cm,
label=90:Damaged myelin] at (m3) {};

\coordinate (wm) at (7.2,-0.5);
\coordinate (wml) at (6.9,-2.2);

%\draw[red] (wm) circle (3pt);
%\draw[red] (wml) circle (3pt);

\draw[gray,thick] (wm)--(healthy);
\draw[gray,thick] (wml)--(damaged);

%\draw[step=1cm,gray,very thin] (-5,2) grid (10,-6);

\end{tikzpicture}

\caption[Demyelination in MS]{Due to the break down of the blood brain barrier,
the body's own immune system attacks the myelin sheaths of the axons, which
causes the formation of demyelinating lesions, visible primarily in the white
matter on conventional \gls{mri} scans. Lesions are highlighted in white.}
\label{fig:ms}
\end{figure}
 
The clinical presentation of MS is very heterogeneous due to the wide range of
areas of the brain and spinal cord that can be affected. Characteristic but not
specific physical signs and symptoms include the loss of sensitivity or changes
in sensation such as tingling or numbness, which typically starts in the fingers
and toes, as well as muscle weakness, difficulty in moving, difficulties with
coordination and balance, problems with speech or swallowing, visual problems,
feeling tired, acute or chronic pain, and bladder and bowel difficulties. In
addition, MS often leads to cognitive problems such as having difficulties
learning and remembering information, or psychiatric and emotional problems such
as depression or frequent mood swings.

The course of the disease is unpredictable. Most patients (around
\SI{80}{\percent}) are initially diagnosed as having \gls{rrms}, which is
characterized by alternating periods of worsening due to inflammatory attacks
and the formation of new lesions, and periods of remission and recovery. The
majority (around \SI{65}{\percent}) of RRMS patients transition to \gls{spms}.
In this stage, the body is no longer able to compensate for the tissue damage,
which leads to the unremitting and progressive accumulation of disability. Other
types of MS include \glslink{ppms}{the primary progressive form (PPMS)},
characterized by the progression of disability from onset with no remissions
after the initial symptoms, and \gls{prms}, which shows progressive accumulation
of disability in addition to clear superimposed relapses.

% TODO: Back this one up with references

\subsection[Measuring disease state and progression]{Measuring Disease State and
Progression}

There is currently no cure for MS. Existing therapies that focus on symptomatic
management and prevention of further damage have variable degrees of
effectiveness, although several recent breakthroughs are promising. To monitor
and further our understanding of the disease, many biomarkers have been
developed that allow for the objective measurement of normal and pathogenic
processes, as well as the monitoring of treatment effect. Current MS biomarkers
can be roughly classified into generic-immunogenetic, laboratorial, and imaging
biomarkers \citep{katsavos2013}. For this thesis, we focus on the accurate
measurement of existing lesion-based imaging biomarkers such as lesion volume
and lesion count, and the development of new biomarkers that capture changes in
brain morphology and white matter lesions---two hallmarks of MS pathology.

Accurate segmentation of MS lesions has shown to be a very challenging due to
the large variability in lesion size, shape, intensity, and locations, as well
as the large variability in imaging contrasts produced by scanners used at
different clinical sites. Despite a growing interest in developing fully
automatic lesion segmentation methods, semi-automatic methods are still the
standard in clinical research, although their use is time-consuming,
laborious, and potentially biasing. It is therefore highly desirable to develop
a fully-automatic lesion segmentation method that is robust to large
variability, while still being able to segment lesions with high accuracy and
sensitivity.

%
% - biomarkers based on the accurate delineation of lesions such as lesion count
%   and lesion volume
% - To measure those changes accurately, accurate segmentations of different
%   tissue types and lesions are required.
% - A lot of progress has been made to accurately separate the brain from
%   non-brain tissues (brain extraction) and segmenting the brain into its three
%   main tissue types, CSF, WM, GM.
% - Segmenting lesions very challenging due to the large variability in lesion
%   size, shape, contrast, and location as well the the large variability in
%   normal tissue contrasts produced by different scanners at different clinical
%   cites.
% - There is still no fully automatic and robust method for the segmentation of
%   lesions that is robust and accurate enough to be used in routine clinical
%   practice.

Imaging biomarkers used in clinical trials mostly focus on volumetric measures
of global and local changes, which are important and relatively easy to compute,
but only correlate modestly with clinical scores. One reason for the modest
correlation is that they do not reflect potentially important structural
variations, such as shape changes in the brain and the spatial dispersion of
lesions. Therefore, it would be highly desirable to develop biomarkers that
capture potentially important patterns of the variability in brain morphology
and lesion distribution, which would advance our understanding of the complex
pathology of MS.

% TODO: Most of these points should go into the discussion section of why the
% correlations are still not as high as we would wish for.
%
% they only correlate modestly with clinical
% disability scores, which limits their utility for the purpose of personalized
% medicine. There are a number of key reasons why the current imaging biomarkers
% do not have stronger quantitative relationships with clinical scores:
% \begin{itemize}
% 
% \item Due to the wide range of symptoms, MS disability is difficult to score
% comprehensively in routine clinical practice. For example, the Kurtzke expanded
% disability status scale (EDSS)~\citep{Kurtzke:1983} is the most commonly used
% clinical score, but does not account for cognitive impairment, which is a
% significant contributor to disability in the majority of MS
% patients~\citep{Chiaravalloti:2008}.
% 
% \item Through neuroplasticity, the brain and spinal cord can adapt to damage in
% order to maintain functionality~\citep{Tomassini:2012}. As a result, clinically
% silent or subtle pathology is often present.
% 
% \item Conventional MRI does not capture all aspects of MS pathology. For
% example, white matter that appears normal on conventional MRI may actually have
% reduced myelin~\citep{Laule:2004}, a nerve insulator that is critical for proper
% signal conduction. Demyelination is a key pathological feature of MS.
% 
% \item The current established imaging biomarkers largely capture volumetric
% changes, which are important and relatively easy to compute, but do not reflect
% potentially important structural variations, such as shape changes in the brain
% and spatial dispersion of the lesions.
% 
% \end{itemize}
% It would be highly desirable to develop biomarkers that capture potentially
% important patterns of variability in brain morphology and lesion distribution,
% which would advance our understanding of the complex pathology of MS.

% An alternative to using biomarkers for monitoring MS is the use of clinical
% scores. In contrast to biomarkers, clinical scores measure the impact of the
% disease on the patient's life directly through clinical tests. The two most
% widely used scores of measuring disability in MS are the expanded disability
% status scale (EDSS) and the multiple sclerosis function composite. The EDSS
% provides a discrete scale of impairment to ambulation and ranges from 0 (normal
% neurological exam) to 10 (death due to MS). Despite its wide use in clinical
% trials, EDSS have been criticized for putting too much emphasise on the motor
% function of the patient, while ignoring other areas of the patient's life such
% as cognitive functions and upper body movement. The MSFC is a composite score
% composed of the following three subscores: 1) The timed 25 feet walk (T25W)
% measures impairment of the lower limbs, 2) the 9-hole peg test (9-HPT) measures
% impairment of the upper limbs, and 3) the paced auditory serial addition test
% (PASAT) measures impairment of the cognitive function of a patient.
% 
% - Scores don't correlate well with biomarkers
% - Why is this so?
% - Need better biomarkers
% - Maybe introduce clinical scores first then say biomarkers can be used as well
% - Difficulties measuring them
% - Difficulties with correlation to clinical scores

\section[Proposed method]{Proposed Method}

\subsection{Objectives}

% More general and taylored to the challenges. Robust method is the objective.
% Contribution is the development of a method based on neural networks and the
% contributions in there.

The clinical motivation of the thesis is to develop methods that facilitate the
automatic measurement of MS disease state and progression that are visible on
conventional structural MRIs. To that end, we have identified two key
applications. One is the development of a fully automatic lesion segmentation
method that is able to segment lesions over a large range of sizes and in the
presence of varying imaging contrasts and imaging artifacts produced by
different scanners, which would allow for the accurate measurement of
lesion-based imaging biomarkers such as lesion load and lesion count. The other
key application is the development of a method that automatically discovers
potentially important patterns of variability in brain morphology and lesion
distribution, with the goal to derive new imaging biomarkers that correlate
stronger with the clinical measures of MS disability than traditionally used
volumetric measures. The global objective of the thesis is to determine the
capabilities of deep learning for these two clinical applications.

\subsection{Overview}

The two methods developed for segmenting MS lesions and modelling patterns of
variability are both based on deep learning, a field within machine learning
that is inspired by the learning capabilities of the brain. The human brain has
often served as a model for computer vision algorithms. For example, SIFT
features \citep{lowe1999}, inspired by neurons of the inferior temporal cortex,
have proved to be very robust for object recognition. Resembling the receptive
field of simple cells of the primary visual cortex, 2D Gabor filters have been
used to describe textures for segmentation \citep{grigorescu2002}. In contrast,
deep learning algorithms try to mimic how the visual system learns instead of
copying what it has learned. First evidence for the learning capabilities of the
visual system were found by \citep{wiesel1963}, who investigated the visual
cortex of cats. They showed that the receptive fields of neurons are learned
from a continuous stream of images early in the development of the visual system
\citep{wiesel1963}, but they are also fine-tuned later \citep{karni1991}.
This allows the neurons of the visual cortex to adapt to the type of images to
which it is exposed. While it is difficult, for example, for an average person
to recognize the differences between cows of the same breed, the feature
detecting neurons of the visual system of cattle farmers are highly tuned to
their appearance, allowing them to recognize individual cows easily. This
suggests that a learning-based model for classification should not only learn to
perform the requested task based on a set of pre-defined features, but also
learn the features that are most suitable to perform the task. The joint
learning of feature extraction and prediction, also known as end-to-end
learning, is possible through the use of deep learning methods, which use
multiple layers of nonlinear processing units to learn a feature hierarchy
directly from the input data without a dedicated feature extraction step.

Deep learning has successfully been used in many research areas such as object
recognition \citep{krizhevsky2012}, speech recognition \citep{hinton2012deep},
natural language understanding \citep{collobert2011natural}, and language
translation \citep{sutskever2014sequence}. Deep learning methods are
particularly successful due to their ability to recognize complex and highly
nonlinear patterns in large amounts of training data, which facilitates the
learning of models that are robust to large variability. This motivates the use
of deep learning methods for segmenting MS lesions, because the large
variability in lesion shape, size, contrast, and location as well as changes in
imaging contrasts produced by different \gls{mri} scanners make lesion
segmentation challenging. Beyond voxel classification, deep learning methods can
also be used to model highly nonlinear patterns of variability in groups of
images. This allows deep learning models such as deep belief networks to be used
for manifold learning, e.g., of hand written digits \citep{hinton2006b} or, as
we will show in \ref{sec:manifold}, brain MRIs \citep{brosch2013}. However,
deep learning algorithms as implemented by widely used deep learning frameworks
were originally developed for application to small 2D images and do not scale
well to large 3D volumes in terms of training time and memory requirements,
which prevents the use of out-of-the-box implementations for 3D medical image
analysis.

\subsection{Contributions}

In the course of developing deep learning-based methods for MS lesion
segmentation and pattern discovery, we have made the following main
contributions:
\begin{enumerate}
\item We have developed a novel training algorithm for convolutional deep belief
networks and convolutional neural networks that performs training in the
frequency domain. The speed-ups gained by our method compared to state-of-the
art spatial domain implementations and the reduced memory requirements compared
to other frequency domain methods enable the application of deep learning to
high-resolution 3D medical images.
  
\item We have developed a neural network architecture that jointly learns
features at different scales that are tuned to segmenting MS lesions and
performs the segmentation based on the automatically learned features. The joint
learning of feature extractor and classifier facilitates the learning of
features that are robust to the large variability of MS lesions and varying
contrasts produced by different scanners.

% \item In contrast to previous patch-based deep learning segmentation approaches,
% our network feeds through entire volumes, which removes the need to select
% representative patches, eliminates redundant calculations where patches overlap,
% and therefore scales up more efficiently with image resolution.

% There are many more contributions for segmentation than manifold
% learning; this should be more balanced. I suggest removing some the
% segmentation contributions so that you will have about 5 total contributions.
% As discussed, a main contribution in manifold learning is that you showed that
% deep learning can be applied to manifold learning of brain MRIs. Another is
% that joint learning can be performed to model key pathology features together,
% such as lesions and morphology.

% \item In contrast to previous fully convolutional deep learning segmentation
% approaches, our method produces segmentations of the same resolution and scale
% as the input images, independent of the parameters of the network architecture,
% which enables the use of deeper networks without requiring special handling
% of the border regions.

\item We have developed a novel objective function for training neural networks
that is suitable for the classification of vastly unbalanced classes, such as
the segmentation of MS lesions, which typically comprise less than one percent
of the image.

% \item Our proposed network architecture contains shortcut connections between
% the feature extraction and segmentation pathways, which allows for the learning
% of features at different scales and thereby facilitates the accurate
% segmentation of MS lesions across a wide range of sizes.

\item This is the first work to demonstrate that deep learning can be applied to
manifold learning of brain MRIs.

\item We have developed a framework for modelling changes in brain morphology
and lesion distribution with only a few parameters, which also show improved
correlation with clinical scores compared to established volumetric imaging
biomarkers.

% \item In contrast to previous manifold learning approaches, our method does not
% assume the ambient space to be locally linear and also does not require the
% definition of a suitable similarity measure, or building a proximity graph,
% which is particularly advantageous for modelling lesions, because their sparse
% and random nature makes defining a suitable distance measure between lesion
% images challenging.
\end{enumerate}

\section[Thesis outline]{Thesis Outline}

The rest of this thesis is organized into five chapters as outlined below:

\subsection*{\ref{sec:background}---\nameref{sec:background}}

In this chapter, we will briefly introduce the supervised and unsupervised deep
learning models that form the basis for the lesion segmentation and manifold
learning methods, which are discussed further in \ref{sec:segmentation}
and \ref{sec:manifold}. We will start with a description of dense neural
networks \citep{farley1954,werbos1974,rumelhart1986} and convolutional
neural networks \citep{fukushima1980,lecun1989,lecun1998}. In the second
part, we will give a brief overview of restricted Boltzmann Machines
\citep{freund1992,hinton2010a}, which are the building blocks of deep belief
networks \citep{hinton2006b}.

\subsection*{\ref{sec:training}---\nameref{sec:training}}

Deep learning has traditionally been computationally expensive and advances in
training methods have been the prerequisite for improving its efficiency in
order to expand its application to a variety of image classification problems.
In this chapter, we address the problem of efficient training of convolutional
deep belief networks by learning the weights in the frequency domain, which
eliminates the time-consuming calculation of convolutions. An essential
consideration in the design of the algorithm is to minimize the number of
transformations to and from frequency space. We have evaluated the running time
improvements using two standard benchmark data sets, showing a speed-up of up to
8 times on 2D images and up to 200 times on 3D volumes. In addition, we have
directly compared the time required to calculate convolutions using our method
with the \gls{cudnn}, the current state-of-the-art library for calculating 2D
and 3D convolutions, with the results showing that our method can calculate
convolutions up to 20 times faster than cuDNN. Our training algorithm makes
training of convolutional deep belief networks and convolutional neural networks
on 3D volumes with a resolution of up to \num{128x128x128} voxels
practical, which opens new directions for using deep learning for medical image
analysis.

\subsection*{\ref{sec:segmentation}---\nameref{sec:segmentation}}

In this chapter, we present a novel segmentation approach based on deep 3D
convolutional encoder networks with shortcut connections and apply it to the
segmentation of MS lesions in magnetic resonance images. Our model is a neural
network that consists of two interconnected pathways, a convolutional pathway,
which learns increasingly more abstract and higher-level image features, and a
deconvolutional pathway, which predicts the final segmentation at the voxel
level. The joint training of the feature extraction and prediction pathways
allows for the automatic learning of features at different scales that are
optimized for accuracy for any given combination of image types and segmentation
task. In addition, shortcut connections between the two pathways allow high- and
low-level features to be integrated, which enables the segmentation of lesions
across a wide range of sizes. We have evaluated our method on two publicly
available data sets (MICCAI 2008 and ISBI 2015 challenges) with the results
showing that our method performs comparably to the top-ranked state-of-the-art
methods, even when only relatively small data sets are available for training.
In addition, we have compared our method with five freely available and widely
used MS lesion segmentation methods (EMS, LST-LPA, LST-LGA, Lesion-TOADS, and
SLS) on a large data set from an MS clinical trial.
The results show that our method consistently outperforms these other methods
across a wide range of lesion sizes.

\subsection*{\ref{sec:manifold}---\nameref{sec:manifold}}

Manifold learning of medical images plays a potentially important role for
modelling anatomical variability within a population with applications that
include segmentation, registration, and prediction of clinical parameters.
In this chapter, we describe a novel method for learning the manifold of 3D
brain images and for building a statistical model of brain images that can
automatically discover spatial patterns of variability in brain morphology and
lesion distribution. We propose building such a model using a deep belief
network, a layered network whose parameters can be learned from training
images. In contrast to other manifold learning algorithms, this approach does
not require a prebuilt proximity graph, which is particularly advantageous for
modelling lesions, because their sparse and random nature makes defining a
suitable distance measure between lesion images challenging. Our results show
that this model can automatically learn a low-dimensional manifold of brain
volumes that detects modes of variations that correlate to demographic and
disease parameters. Furthermore, our model can automatically discover the
classic patterns of MS pathology, as well as more subtle ones, and the
computed parameters have strong relationships to MS clinical scores.

\subsection*{\ref{sec:conclusions}---\nameref{sec:conclusions}}

This chapter concludes the thesis with a brief summary of the problems addressed
and key results. In addition, some directions for future work are given in the
broader context of deep learning for medical imaging. In particular, we will
give suggestions for new medical image analysis applications of deep learning
and how to deal with relatively small data sets using data augmentation. In
addition, we will discuss two potential advancements of neural networks that we
believe will be particularly important for medical applications.

%   - Use more data, handle variability better, age of large data is perfect for deep learning
%   - No assumptions other than test data should be similarly distributed than training data -> robust and fast
%   - What can we learn from it? Need visualization of what the network is thinking. Probably relearn what the NN is doing similar to how 
%     brain research works: learn from good and bad cases or from deactivating certain parts of the network to infer what it does.
%   - Theory of learning combining neuroscience and computer science

\begin{comment}

Multiple sclerosis (MS) is a chronic, degenerative disease of the brain and
spinal cord. The clinical presentation of MS is very heterogeneous, and the
range and severity of symptoms can vary greatly between patients. The clinical
course of MS is highly unpredictable, but most patients are initially diagnosed
as having relapsing remitting MS (RRMS), which is characterized by inflammatory
attacks separated by variable periods of remission and recovery. The majority of
RRMS patients will eventually transition into the secondary progressive MS
(SPMS) phase, in which there is an unremitting and progressive accumulation of
disability. There is currently no cure for MS. Existing therapies that focus on
symptomatic management and prevention of further damage have variable degrees of
effectiveness, although several recent breakthroughs are promising. MS pathology
originates at the cellular level and many aspects are not well understood, but
there are characteristic (but not specific) signs of tissue damage, the most
recognizable of which are white matter lesions (WMLs) and brain atrophy, or
shrinkage due to degeneration. These signs can be observed on MRI, which has
become a vital tool for non-invasively monitoring MS patients in the clinic and
for advancing the understanding of MS pathology. WML counts and volume and brain
volume have become established imaging biomarkers for MS clinical trials, and
there is promise for their use in routine clinical practice, but they generally
only correlate modestly with clinical disability scores. The weak link between
the image-based measures of MS pathology and disability scores is known as the
``clinico-radiological paradox'' of MS, and results in the low utility of
current imaging biomarkers for the purposes of personalized medicine. There are
a number of key reasons why the current imaging biomarkers do not have stronger
quantitative relationships with clinical scores:

\begin{itemize}

\item Due to the wide range of symptoms, MS disability is difficult to score
comprehensively in routine clinical practice. For example, the Kurtzke expanded
disability status scale (EDSS)~\citep{Kurtzke:1983} is the most commonly used
clinical score, but does not account for cognitive impairment, which is a
significant contributor to disability in the majority of MS
patients~\citep{Chiaravalloti:2008}.

\item Through neuroplasticity, the brain and spinal cord can adapt to damage in
order to maintain functionality~\citep{Tomassini:2012}. As a result, clinically
silent or subtle pathology is often present.

\item Conventional MRI does not capture all aspects of MS pathology. For
example, white matter that appears normal on conventional MRI may actually have
reduced myelin~\citep{Laule:2004}, a nerve insulator that is critical for proper
signal conduction. Demyelination is a key pathological feature of MS.

\item The current established imaging biomarkers largely capture volumetric
changes, which are important and relatively easy to compute, but do not reflect
potentially important structural variations, such as shape changes in the brain
and spatial dispersion of the lesions.

\item Traditional statistical approaches place a strong emphasis on
interpretability, often at the sacrifice of accuracy, and simple prediction
models such as logistic regression are common. The general assumption is that
the data is generated by a known stochastic data model, with the goodness-of-fit
evaluated by residual analysis, but this works well only with a very low number
of variables~\citep{Breiman2001}.

\end{itemize}

% Add to manifold introduction
Changes in brain morphology and white matter lesions are two hallmarks of MS
pathology, but their variability beyond volumetrics is poorly characterized. To
further the understanding of complex MS pathology,

% Add to segmentation introduction
Focal lesions in the brain and spinal cord are one of the hallmarks of MS
pathology, and are primarily visible in the white matter on structural MRIs.
These lesions are observable as hyperintensities on T2w, proton density-weighted
(PDw), or fluid-attenuated inversion recovery (FLAIR) scans, and as
hypointensities, or ``black holes'' on T1w scans. Imaging biomarkers based on
the identification of lesions, such as lesion count and lesion volume, have
established their importance for assessing disease progression and treatment
effect. However, lesions vary greatly in size, shape, intensity and location,
which makes their automatic and accurate segmentation challenging.

% See literature review as a tool, not as a goal. Try to structure the thesis
% without it and put it where it belongs. Maybe restructure later. I needed a part
% of it already for the motivation. I will put a lot in it when I describe the
% methods. An I will put something in there, when I describe the applications.

\end{comment}