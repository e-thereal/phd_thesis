\begin{tikzpicture}[%
scale=0.85,
x  = {(1cm,0cm)},
y  = {(0.33cm,0.23cm)},
z  = {(0cm,1cm)}
%z  = {(0.9cm,-0.1cm)},
%x  = {(0.33cm,0.23cm)},
%y  = {(0cm,1cm)}
]

\tikzstyle{every node}=[font=\small, inner sep=3pt, align=center]
\tikzstyle{every pin}=[align=center,fill=white]
\tikzstyle{dbnlabel}=[font=\sffamily\normalsize]

\tikzstyle{image}=[fill=white, fill opacity=0.75]
\tikzstyle{pinline}=[thin, black!35]
\tikzstyle{kernelline}=[very thin]


%%%%%%%%%%%%%%%%%         
% INPUT LAYER
%%%%%%%%%%%%%%%%%

\foreach \x in {1, ..., 3} {
\def\y{0.25*\x}
\draw[image] (\y, -2,-2) coordinate (A1\x) -- (\y, 2,-2)
coordinate (B1\x) -- (\y, 2,2) coordinate (C1\x) -- (\y, -2,2) -- cycle;

\draw[kernelline]
      (\y, -1.6, 1.6) \ifnum\x=3 \else -- +(0.25, 0, 0) +(0,0,0) \fi
   -- (\y, -0.8, 1.6) \ifnum\x=3 \else -- +(0.25, 0, 0) +(0,0,0) \fi
   -- (\y, -0.8, 0.8) \ifnum\x=3 \else -- +(0.25, 0, 0) +(0,0,0) \fi
   -- (\y, -1.6, 0.8) \ifnum\x=3 \else -- +(0.25, 0, 0) +(0,0,0) \fi
   -- (\y, -1.6, 1.6);
   ;
\ifnum\x=3
\draw[kernelline] (\y, -1.6, 1.6) coordinate(a1) -- (1.65, -1.2, 1.2);
\draw[kernelline] (\y, -1.6, 0.8) coordinate(b1) -- (1.65, -1.2, 1.2);
\draw[kernelline] (\y, -0.8, 0.8) coordinate(c1) -- (1.65, -1.2, 1.2);
\draw[kernelline] (\y, -0.8, 1.6) coordinate(d1) -- (1.65, -1.2, 1.2);
\fi
}

%%%%%%%%%%%%%%%%%         
% HIDDEN LAYER
%%%%%%%%%%%%%%%%%

\foreach \x in {0, ..., 7} {
\def\y{0.125*\x+1.65}
\draw[image] (\y,-1.6,-1.6) coordinate(A2\x) -- (\y,1.6,-1.6)
-- (\y,1.6,1.6) coordinate (C2\x) -- (\y,-1.6,1.6) coordinate(D2\x) -- cycle;

\draw[kernelline]
      (\y, -0.8, 0.8) \ifnum\x=7 \else -- +(0.125, 0, 0) +(0,0) \fi
   -- (\y, -0.4, 0.8) \ifnum\x=7 \else -- +(0.125, 0, 0) +(0,0) \fi
   -- (\y, -0.4, 0.4) \ifnum\x=7 \else -- +(0.125, 0, 0) +(0,0) \fi
   -- (\y, -0.8, 0.4) \ifnum\x=7 \else -- +(0.125, 0, 0) +(0,0) \fi
   -- (\y, -0.8, 0.8);
\ifnum\x=7
\draw[kernelline] (\y, -0.8, 0.8) coordinate(a2) -- (3.425, -0.6, 0.6);
\draw[kernelline] (\y, -0.8, 0.4) coordinate(b2) -- (3.425, -0.6, 0.6);
\draw[kernelline] (\y, -0.4, 0.4) coordinate(c2) -- (3.425, -0.6, 0.6);
\draw[kernelline] (\y, -0.4, 0.8) coordinate(d2) -- (3.425, -0.6, 0.6);
\fi 
}

\draw[decorate,decoration={brace,raise=20pt}] (A11|-C11) --node[above=25pt]
{Convolutional layer} (C27|-C11);

\foreach \x in {0, ..., 7} {
\def\y{0.125*\x+3.425}
\draw[image] (\y,-0.8,-0.8) coordinate(A3\x) -- (\y,0.8,-0.8)
-- (\y,0.8,0.8) coordinate (C3\x) -- (\y,-0.8,0.8) -- cycle;

\draw[kernelline]
      (\y, -0.6, +0.6) \ifnum\x=7 \else -- +(0.125, 0, 0) +(0,0) \fi
   -- (\y, +0.2, +0.6) \ifnum\x=7 \else -- +(0.125, 0, 0) +(0,0) \fi
   -- (\y, +0.2, -0.2) \ifnum\x=7 \else -- +(0.125, 0, 0) +(0,0) \fi
   -- (\y, -0.6, -0.2) \ifnum\x=7 \else -- +(0.125, 0, 0) +(0,0) \fi
   -- (\y, -0.6, +0.6);
\ifnum\x=7
\draw[kernelline] (\y, -0.6, +0.6) -- (5.2, -0.2, 0.2);
\draw[kernelline] (\y, -0.6, -0.2) -- (5.2, -0.2, 0.2);
\draw[kernelline] (\y, +0.2, -0.2) coordinate (g7) -- (5.2, -0.2, 0.2);
\draw[kernelline] (\y, +0.2, +0.6) -- (5.2, -0.2, 0.2);
\fi 
}

\draw[decorate,decoration={brace,raise=10pt}] (C21|-C21) --node[above=15pt]
{Pooling layer} (C37|-C21);

\foreach \x in {0, ..., 15} {
\def\y{0.125*\x+5.2}
\draw[image] (\y,-0.4,-0.4) coordinate (A4\x) -- (\y,0.4,-0.4)
-- (\y,0.4,0.4) coordinate (C4\x) -- (\y,-0.4,0.4) -- cycle;
}

\begin{scope}[xshift=9cm]
\foreach \x in {1,...,7} {
  \node[circle, draw] (v\x) at (0,0, 0.5*\x - 0.5*4) {};
}
\end{scope}

\draw[shorten >=10pt,shorten <=7pt,dashed] (A415-|C415)--(v1.south west);
\draw[shorten >=10pt,shorten <=7pt,dashed] (C415)--
node[label=120:Vectorize\\ hidden units] {} (v7.north west);

\begin{scope}[xshift=10.5cm]
%\foreach \x/\y in {1/-1.5, 2/-1, 3/-0.5, 4/0, 5/0.5, 6/1, 7/1.5} {
\foreach \x in {1,...,5} {
  \node[circle, draw] (h\x) at (0,0, 0.5*\x - 0.5*3) {};
}
\end{scope}

\foreach \x in {1,...,7} {
	\foreach \y in {1,...,5} {
		\draw (v\x)--(h\y);
	}
}

\draw[decorate,decoration={brace,raise=10pt}] (v7.west|-C21)
 --node[above=15pt] {Dense layer} (h1.east|-C21);
 
\node[above=15pt, fill=white,xshift=-25pt,inner sep=3pt, fill opacity=1, text
opacity=1] 
(convl) at (d1) {convolution};
\draw[pinline, shorten >=4pt, shorten <=2pt] (convl) -- (d1);

\node[above=15pt,fill=white,xshift=35pt,inner sep=3pt]
(pooll) at (d2) {pooling};
\draw[pinline, shorten >=4pt, shorten <=2pt] (pooll) -- (d2);

% Label units

\node[below=10pt,align=center] (input) at (A12) {Multimodal\\ input images};

\draw[pinline, shorten >=4pt] (input) -- (A11);
\draw[pinline, shorten >=4pt] (input) -- (A12);
\draw[pinline, shorten >=4pt] (input) -- (A13);

\node[below=40pt,xshift=20pt,inner sep=5pt] (hiddens) at (A31) {Hidden units};
\draw[pinline, shorten >=4pt] (hiddens) -- (A27);
\draw[pinline, shorten >=4pt] (hiddens) -- (A34);
\draw[pinline, shorten >=4pt] (hiddens) -- (A47);
\draw[pinline, shorten >=4pt] (hiddens) -- (v1);

\node[below=25pt,align=center] (output) at (h1) {Output\\ units};
\draw[pinline, shorten >=4pt] (output) -- (h1);

\begin{comment}

%%%%%%%%%%%%%%%%%         
% OUTPUT LAYER
%%%%%%%%%%%%%%%%%

\begin{scope}[canvas is xy plane at z=3.425]
\draw[image] (-2,-2) coordinate (A) -- (2,-2)
-- (2,2) coordinate (C) -- (-2,2) -- cycle;
\end{scope}

\node[xshift=-0.7cm, left] (lesion) at (A) {$S$};
\draw[pinline, shorten >=4pt] (lesion) -- (A);

\node[xshift=0.7cm, right] (y2) at (C) {$y^{(2)}$};
\draw[pinline, shorten <=4pt] (C) -- (y2);

\draw[decorate,decoration={brace,raise=75pt,mirror}] (B1-|C) --
node[right=80pt] {Convolutional\\ network} (G7-|C1);

\draw[decorate,decoration={brace,raise=95pt,mirror}] (G0-|C1) --
node[right=100pt] {Deconvolutional\\ network} (C);

\end{comment}

\end{tikzpicture}